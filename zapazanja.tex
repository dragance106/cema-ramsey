Da aktiviras virtual environment iz terminala:
Odes u direktorijum gde se nalazi projekat, pa onda

  source venv/bin/activate
  
Posle toga, medjutim, moras da koristis python iz tog virtual environment-a,
tako da program pokreces sa:

  venv/bin/python cema_train_simple_graphs.py
  
Na kraju, kad zavrsis sa virtual environment-om, 
ili kazes

  deactivate
  
ili jednostavno zatvoris prozor terminala.

Da ti se kompjuter ne uspava dok traje treniranje:

  caffeinate -disu

TensorBoard za prikupljanje rezultata startujes pomocu

  /Users/ds/miniforge3/bin/tensorboard --logdir runs

ili gde je vec instaliran tensorboard na lokalnom kompjuteru,
pri cemu je runs poddirektorijum trenutnog direktorijuma sa "cema for graphs" projektom.
Web interfejsu za TensorBoard pristupas iz web browsera pomocu localhost:6006.

Za rad sa jar fajlovima:

  jar tf imejarfajla

daje spisak svih klasa zapakovanih u imejarfajla, dok

  javap -classpath imejarfajla graph6java.Graph

daje spisak svih metoda implementiranih u klasi graph6java.Graph unutar imejarfajla.
